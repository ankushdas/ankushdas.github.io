\documentclass{article}

% Language setting
% Replace `english' with e.g. `spanish' to change the document language
\usepackage[english]{babel}

% Set page size and margins
% Replace `letterpaper' with`a4paper' for UK/EU standard size
\usepackage[letterpaper,top=2cm,bottom=2cm,left=3cm,right=3cm,marginparwidth=1.75cm]{geometry}

% Useful packages
\usepackage{amsmath}
\usepackage{amssymb}
\usepackage{graphicx}
\usepackage{mathpartir}
\usepackage{listings}
\lstset{basicstyle=\ttfamily}
\usepackage{courier}
\usepackage[colorlinks=true, allcolors=blue]{hyperref}

\title{CS 599 D1: Assignment 2}
\author{YOUR NAME, BU ID}
\date{}

\newcommand{\lc}{$\lambda$-calculus}

\newcommand{\m}[1]{\mathsf{#1}}
\newcommand{\elam}[2]{\lambda{#1}.\,{#2}}
\newcommand{\eapp}[2]{#1 \; #2}
\newcommand{\step}{\mapsto}
\newcommand{\val}[1]{#1 \; \m{value}}
\newcommand{\eval}{\Downarrow}
\newcommand{\num}[1]{\overline{#1}}
\newcommand{\numf}[1]{\underline{#1}}
\newcommand{\eif}[3]{\m{if} \; #1 \; \m{then} \; #2 \; \m{else} \; #3}
\newcommand{\multval}[1]{#1 \; \m{mult{-}value}}
\newcommand{\closed}[1]{#1 \; \m{closed}}
\newcommand{\tint}{\m{int}}
\newcommand{\tbool}{\m{bool}}
\newcommand{\elet}[3]{\m{let} \; #1 = #2 \; \m{in} \; #3}
\newcommand{\G}{\Gamma}
\newcommand{\R}[1]{\textcolor{red}{#1}}
\newcommand{\tfloat}{\m{float}}
\newcommand{\emptybox}{\boxed{\textcolor{white}{QWERTY}}}
\newcommand{\mstep}{\step^{*}}
\newcommand{\zero}{\m{zero}}
\renewcommand{\succ}[1]{\m{succ}(#1)}
\newcommand{\semi}{\,;\,}
\newcommand{\natrec}[5]{\m{natrec}(#1 \semi #2 \semi #3.\, #4. \, #5)}
\newcommand{\tnat}{\m{nat}}
\newcommand{\Red}{\m{Red}}
\newcommand{\natred}[1]{#1 \downarrow}
\newcommand{\issubst}{\Vdash}

\newtheorem{problem}{Problem}
\newtheorem{theorem}{Theorem}
\newtheorem{lemma}{Lemma}
\newtheorem{definition}{Definition}
\newenvironment{solution}{\textbf{Solution.}}{}

\begin{document}
\maketitle

\begin{problem}[34 pts]
Solve the following problems for LL1:

    \begin{itemize}
        \item[(4 pts)] Define the rules of the type system for LL1. Although not necessary, but try to use as
        few rules as possible.
        
        \item[(5 pts)] Define the rules of the small-step semantics for LL1. To define the semantics, you can use the $\preceq$
        and $\neg$ operators to compare integer/float values (similar to $\oplus$ for adding integer/float values).

        \item[(5 pts)] Define the rules of the big-step semantics for LL1. Again, you can use the $\preceq$ and $\neg$
        operators to compare integer/float values.
        
        \item[(10 pts)] Is the LL1 language type safe? Either prove the progress and preservation theorems for LL1
        or show a counterexample expression $e$ that violates either the progress or the preservation theorem.

        [Preservation Theorem]:\\
        If $\G \vdash e : \tau$ and $e \step e'$, then $\G \vdash e' : \tau$.

        [Progress Theorem]:\\
        If $\cdot \vdash e : \tau$ then either $e \step e'$ or $\val{e}$.

        \item[(10 pts)] Prove the equivalence of the small-step and big-step semantics. To do this, we need to define
        a new judgment $e \mstep e'$ which is the reflexive transitive closure of $e \step e'$. Formally,
        \begin{mathpar}
            \inferrule*[right=Refl]
            { }
            {e \mstep e}
            \and
            \inferrule*[right=Trans]
            {e_1 \mstep e_2 \and e_2 \mstep e_3}
            {e_1 \mstep e_3}
            \and
            \inferrule*[right=Step]
            {e_1 \step e_2}
            {e_1 \mstep e_2}
        \end{mathpar}

        Write the theorem statement demonstrating the equivalence of the small-step and big-step semantics and then
        prove it.
        
    \end{itemize}
\end{problem}

\begin{solution}

\end{solution}


\begin{problem}[16 pts]
    In this problem, fill in the blanks to make the following typing judgments in \lc{} valid, or briefly explain
    that it is impossible to do so. This might require defining either (i) an expression with a given type,
    or (ii) the typing context for an expression,
    or (iii) the type of a given expression in a typing context,
    or (iv) a combination of the above. (2 pts each)

    \begin{enumerate}
        \item $\cdot \vdash \emptybox : \alpha \rightarrow \alpha$
        
        \item $y : \beta \vdash \emptybox : \alpha \rightarrow \beta$
        
        \item $\cdot \vdash \emptybox : \alpha \rightarrow \beta$
        
        \item $\emptybox \vdash \eapp{x}{x} : \emptybox$
        
        \item $\cdot \vdash \emptybox : \alpha \rightarrow (\alpha \rightarrow \alpha)$
        
        \item $\cdot \vdash \emptybox : (\alpha \rightarrow \alpha) \rightarrow \alpha$
        
        \item $\emptybox \vdash \elam{x}{\eapp{x}{(\eapp{x}{x})}} : \emptybox$
        
        \item $\cdot \vdash \elam{f}{\elam{g}{\elam{x}{\eapp{(\eapp{f}{x})}{(\eapp{g}{x})}}}} : (\alpha \rightarrow \emptybox) \rightarrow (\alpha \rightarrow \emptybox) \rightarrow (\alpha \rightarrow \emptybox)$
        
    \end{enumerate}
\end{problem}

\begin{solution}
  
\end{solution}

\begin{problem}[10 pts]
  Prove closure under head expansion.
\end{problem}

\begin{solution}
  
\end{solution}


\begin{problem}[30 pts]
  Prove the remaining cases of Theorem 3.
  You may state (without proof) lemmas about substitution, but be sure
  to check that they are actually true.
\end{problem}

\begin{solution}
  
\end{solution}


\begin{problem}[20 pts]
    Define the following functions in System T. For each definition below,
    briefly explain the intuition behind your answer (4 pts each).
    You can define and use helper functions to solve these problems.

    \begin{enumerate}
        \item Define $\m{mult}$, where $\m{mult} \; \num{m} \; \num{n} \mstep \num{m \otimes n}$.
        \item Define $\m{minus}$, where $\m{minus} \; \num{m} \; \num{n} \mstep \num{m - n}$ if $\num{m} > \num{n}$. It should produce 0 otherwise.
        \item Define $\m{leq}$, where $\m{leq} \; \num{m} \; \num{n} \mstep \succ{\zero}$ if $\num{m} \leq \num{n}$ and
        $\m{leq} \; \num{m} \; \num{n} \mstep \zero$ otherwise.
        \item Define $\m{mod}$ where $\m{mod} \; \num{m} \; \num{n} = \num{m \; \m{mod} \; n}$. You may pick appropriate defaults when $n = 0$.
        \item Define $\m{cube}$ where $\m{cube} \; \num{n} = \num{n \otimes n \otimes n}$.
    \end{enumerate}

\end{problem}

\begin{solution}
  
\end{solution}

\end{document}