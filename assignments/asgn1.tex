\documentclass{article}

% Language setting
% Replace `english' with e.g. `spanish' to change the document language
\usepackage[english]{babel}

% Set page size and margins
% Replace `letterpaper' with`a4paper' for UK/EU standard size
\usepackage[letterpaper,top=2cm,bottom=2cm,left=3cm,right=3cm,marginparwidth=1.75cm]{geometry}

% Useful packages
\usepackage{amsmath}
\usepackage{graphicx}
\usepackage{mathpartir}
\usepackage[colorlinks=true, allcolors=blue]{hyperref}

\title{CS 599: Assignment 1}
\author{YOUR NAME, BU ID}
\date{}

\newcommand{\lc}{$\lambda$-calculus}

\newcommand{\m}[1]{\mathsf{#1}}
\newcommand{\elam}[2]{\lambda{#1}.{#2}}
\newcommand{\eapp}[2]{#1 \; #2}
\newcommand{\step}{\mapsto}
\newcommand{\val}[1]{#1 \; \m{value}}
\newcommand{\eval}{\Downarrow}
\newcommand{\num}[1]{\overline{#1}}
\newcommand{\eif}[3]{\m{if} \; #1 \; \m{then} \; #2 \; \m{else} \; #3}
\newcommand{\multval}[1]{#1 \; \m{mult{-}value}}

\newtheorem{problem}{Problem}
\newenvironment{solution}{\textbf{Solution.}}{}


\begin{document}
\maketitle

\begin{problem}[5 pts]
    In class, we looked at a simple arithmetic expression language. Similarly, for this assignment, we will study
    a simple boolean expression language. This language contains 3 expressions written using the following syntax:
    \[
        \text{Expressions} \quad e ::= \m{true} \mid \m{false} \mid \eif{e}{e}{e}
    \]
    The first two expressions are just the values true and false respectively. The expression $\eif{b}{e_1}{e_2}$
    describes the standard $\m{if}$ behavior, i.e., if $b$ evaluates to true, then the expression reduces to $e_1$,
    else if $b$ evaluates to false, then the expression reduces to $e_2$.

    \begin{itemize}
        \item[(3 pts)] Write and explain each rule of semantics for this language. Note that the semantics will be written
        using the two standard judgments: $\val{e}$ and $e \step e'$.
        
        \item[(2 pts)] Consider the following expression:
        \[
            \eif{(\eif{\m{true}}{\m{false}}{\m{true}})}{(\eif{\m{false}}{\m{false}}{\m{true}})}{(\eif{\m{true}}{\m{true}}{\m{false}})}
        \]
        Evaluate this expression using the rules of the semantics you just defined above.
    \end{itemize}
\end{problem}

\begin{solution}

\end{solution}


\begin{problem}[16 pts]
    Now, we will try to express booleans using \lc{}. The general technique is to represent the two values, i.e., true and false using
    normal forms (or values) in \lc{}. Furthermore, they should be closed, that is, not contain any free variables. Since we need to
    distinguish between the two values, we need to introduce two variables.
    We then define rather arbitrarily one to be true and the other to be false
    \[
        \m{true} = \elam{x}{\elam{y}{x}} \qquad
        \m{false} = \elam{x}{\elam{y}{y}}
    \]

    \begin{itemize}
        \item[(8 pts)] Our first task is to express the $\m{if}$ expression defined in the previous problem in \lc{}. To do this,
        let's consider two constants, $T$ and $E$. $T$ represents the expression for the `then' branch, while $E$ represents the
        expression for the `else' branch. In other words, construct the expression $\eif{b}{T}{E}$ in \lc{}, i.e., using $\lambda$
        expressions, function applications, and variables.

        \item[(2 pts)] Verify that the expression you constructed is indeed equivalent to $\eif{b}{T}{E}$. To do this, suppose the
        expression you constructed above is $I$. Then, $\eapp{(\elam{b}{I})}{\m{true}}$ should evaluate to $T$.
        And similarly, $\eapp{(\elam{b}{I})}{\m{false}}$ should evaluate to $E$, where $\m{true}$ and $\m{false}$ are the expressions
        above. Evaluate these two expressions using the semantics rules of \lc{} to verify that they indeed evaluate to $T$ and $F$
        respectively.
        
        \item[(6 pts)] Use the expression above to construct the $\m{and}$, $\m{or}$, and $\m{not}$ functions.
        \begin{itemize}
            \item $\m{and}$ should have the form $\elam{b_1}{\elam{b_2}{\ldots}}$. It takes two booleans as parameters and returns
            a boolean.
            \item Similarly, $\m{or}$ should have the same form.
            \item $\m{not}$ should have the form $\elam{b}{\ldots}$, i.e., takes a boolean as a parameter and returns a boolean.
        \end{itemize}
    \end{itemize}
\end{problem}

\begin{solution}
    
\end{solution}


\begin{problem}[4 pts]
    Consider the following expression, popularly known as the Y-combinator, defined as
    $\eapp{(\elam{x}{\eapp{x}{x}})}{(\elam{x}{\eapp{x}{x}})}$.
    What does this expression evaluate to? Show each step of the evaluation. Is this expression a value or not?
    Does this violate the progress theorem we discussed in the lecture?
\end{problem}

\begin{solution}
    
\end{solution}


\begin{problem}[25 pts]
    Let's return to the LL1 language discussed in the lecture. We now add an expression for multiplication in the language,
    so the formal syntax is
    \[
        \text{Expressions} \quad e ::= \num{n} \mid e + e \mid e \times e
    \]

    \begin{itemize}
        \item[(5 pts)] First, define the rules of (small-step) semantics for this language.
        
        \item[(5 pts)] You will notice there is some non-determinism in the semantics rules, i.e., for a given expression $e$,
        it will evaluate to two different values based on the order in which the rules are applied. Give an example of such an
        expression $e$ and show that it evaluates to two different values in two different derivations.
        Again, show each step of the derivation.

        \item[(10 pts)] Our goal is to eliminate this non-determinism. For this, we have to come up with evaluation precedence,
        i.e., the order in which operators are applied. We arbitrarily decide that multiplication has higher precedence than
        addition, i.e., \textbf{all multiplication must occur before any addition}. Define the semantics rules for such an evaluation.
        \textbf{Hint:} It might help to define an auxiliary judgment: $\multval{e}$, which holds iff expression $e$ contains
        no addition operators.

        \item[(5 pts)] Take the same expression $e$ defined in sub-task 2 and evaluate it using the semantics rules defined
        in sub-task 3. Is there still any non-determinism in these rules? Can $e$ still evaluate to different values?
    \end{itemize}
\end{problem}

\begin{solution}

\end{solution}





\end{document}