\documentclass{article}

% Language setting
% Replace `english' with e.g. `spanish' to change the document language
\usepackage[english]{babel}

% Set page size and margins
% Replace `letterpaper' with`a4paper' for UK/EU standard size
\usepackage[letterpaper,top=2cm,bottom=2cm,left=3cm,right=3cm,marginparwidth=1.75cm]{geometry}

% Useful packages
\usepackage{amsmath}
\usepackage{amssymb}
\usepackage{graphicx}
\usepackage{mathpartir}
\usepackage{listings}
\lstset{basicstyle=\ttfamily}
\usepackage{courier}
\usepackage[colorlinks=true, allcolors=blue]{hyperref}

\title{\vspace{-2em}
CS 599 A1: Assignment 3}
\author{YOUR NAME, BU ID}
\date{}

\newcommand{\lc}{$\lambda$-calculus}

\newcommand{\m}[1]{\mathsf{#1}}
\newcommand{\mi}[1]{\mathit{#1}}
\newcommand{\mb}[1]{\mathbf{#1}}
\newcommand{\mt}[1]{\mathtt{#1}}
\newcommand{\elam}[2]{\lambda{#1}.\,{#2}}
\newcommand{\eapp}[2]{#1 \; #2}
\newcommand{\step}{\mapsto}
\newcommand{\val}[1]{#1 \; \m{value}}
\newcommand{\eval}{\Downarrow}
\newcommand{\num}[1]{\overline{#1}}
\newcommand{\numf}[1]{\underline{#1}}
\newcommand{\eif}[3]{\m{if} \; #1 \; \m{then} \; #2 \; \m{else} \; #3}
\newcommand{\multval}[1]{#1 \; \m{mult{-}value}}
\newcommand{\closed}[1]{#1 \; \m{closed}}
\newcommand{\tint}{\m{int}}
\newcommand{\tbool}{\m{bool}}
\newcommand{\elet}[3]{\m{let} \; #1 = #2 \; \m{in} \; #3}
\newcommand{\G}{\Gamma}
\newcommand{\R}[1]{\textcolor{red}{#1}}
\newcommand{\tfloat}{\m{float}}
\newcommand{\emptybox}{\boxed{\textcolor{white}{QWERTY}}}
\newcommand{\mstep}{\step^{*}}
\newcommand{\zero}{\m{zero}}
\renewcommand{\succ}[1]{\m{succ}(#1)}
\newcommand{\semi}{\,;\,}
\newcommand{\natrec}[5]{\m{natrec}(#1 \semi #2 \semi #3.\, #4. \, #5)}
\newcommand{\tnat}{\m{nat}}
\newcommand{\Red}{\m{Red}}
\newcommand{\natred}[1]{#1 \downarrow}
\newcommand{\issubst}{\Vdash}
\newcommand{\ift}{\m{int{-}or{-}float}}
\newcommand{\ecase}[3]{\m{case} \; #1 \; (#2 \Rightarrow #3)}
\newcommand{\ecaselr}[3]{\m{case} \; #1 \; (\m{inl} \Rightarrow #2 \mid \m{inr} \Rightarrow #3)}
\newcommand{\case}[2]{\m{case} \; #1 \; (#2)}
\newcommand{\erecv}[2]{#2 \leftarrow \m{recv} \; #1}
\newcommand{\esend}[2]{\m{send} \; #1 \; #2}
\newcommand{\esendl}[2]{#1.#2}
\newcommand{\ewait}[1]{\m{wait} \; #1}
\newcommand{\eclose}[1]{\m{close} \; #1}
\newcommand{\fwd}[2]{#1 \leftrightarrow #2}
\newcommand{\espawn}[3]{#1 \leftarrow #2 \; #3}
\newcommand{\ichoice}[1]{\oplus\{#1\}}
\newcommand{\echoice}[1]{\&\{#1\}}
\newcommand{\with}{\,\&\,}
\newcommand{\one}{\mathbf{1}}
\newcommand{\tensor}{\otimes}
\newcommand{\lolli}{\multimap}
\newcommand{\config}{\mathcal{C}}
\newcommand{\dc}{\mathcal{D}}
\newcommand{\D}{\Delta}
\newcommand{\T}{\Theta}
\newcommand{\poised}[1]{#1 \; \m{poised}}
\newcommand{\proc}[2]{\m{proc}(#1, #2)}
\newcommand{\msg}[2]{\m{msg}(#1, #2)}
\newcommand{\fresh}[1]{(#1 \; \m{fresh})}

\newcommand{\cons}{\mathcal{C}}
\newcommand{\vars}{\mathcal{V}}
\newcommand{\proves}{\vDash}

\newcommand{\tforall}[1]{\forall #1. \, }
\newcommand{\texists}[1]{\exists #1. \, }
\newcommand{\tassertop}{{?}}  % -fp was: ?
\newcommand{\tassumeop}{{!}}  % -fp was: !
\newcommand{\tassert}[1]{\tassertop\{#1\}. \,} % -fp was: \; \tassertop
\newcommand{\tassume}[1]{\tassumeop\{#1\}. \,} % -fp was: \; \tassumeop

\newcommand{\esendn}[2]{\m{send} \; #1 \; \{#2\}}
\newcommand{\erecvn}[2]{\{#2\} \leftarrow \m{recv} \; #1}
\newcommand{\eassume}[2]{\m{assume} \; #1 \; \{#2\}}
\newcommand{\eassert}[2]{\m{assert} \; #1 \; \{#2\}}
\newcommand{\eimposs}{\m{impossible}}

\newcommand{\epair}[2]{\langle #1, #2 \rangle}
\newcommand{\letpair}[4]{\m{let} \; \langle #1, #2 \rangle = #3 \; \m{in} \; #4}
\newcommand{\inl}[1]{\m{inl}(#1)}
\newcommand{\inr}[1]{\m{inr}(#1)}
\newcommand{\match}[5]{\m{match} \; #1 \; \m{with} \; \{\,\m{inl}(#2) \to #3 \mid \m{inr}(#4) \to #5\,\}}
\newcommand{\letapp}[4]{\m{let} \; #1 = #2 \; #3 \; \m{in} \; #4}

\newtheorem{problem}{Problem}
\newtheorem{theorem}{Theorem}
\newtheorem{lemma}{Lemma}
\newtheorem{definition}{Definition}
\newenvironment{solution}{\textbf{Solution.}}{}


\begin{document}
\maketitle

\section*{Derivations and Programs}
This assignment will be all about writing derivations and their corresponding programs.
For your convenience, I have provided the typing rules of both languages.

\subsection*{Constructive Logic}
\paragraph*{Syntax}
\begin{align*}
  \text{Expressions} \quad e & ::= \inl{e} \mid \inr{e} \mid \match{x}{y}{e}{z}{e} \mid \epair{e}{e} \mid \letpair{y}{z}{x}{e} \\
  & \;\; \mid \;\; \elam{x}{e} \mid \letapp{y}{f}{e}{e} \\
  \text{Types} \quad \tau & ::= \tau \lor \tau \mid \tau \land \tau \mid \tau \Rightarrow \tau
\end{align*}

\paragraph*{Typing Rules}
  \begin{mathpar}
  \inferrule*[right=$\lor$R$_1$]
  {\G \vdash e : \tau_1}
  {\G \vdash \inl{e} : \tau_1 \lor \tau_2}
  \and
  \inferrule*[right=$\lor$R$_2$]
  {\G \vdash e : \tau_2}
  {\G \vdash \inr{e} : \tau_1 \lor \tau_2}
  \and
  \inferrule*[right=$\lor$L]
  {\G, y : \tau_1 \vdash e_1 : \tau \and \G, z : \tau_2 \vdash e_2 : \tau}
  {\G, x : \tau_1 \lor \tau_2 \vdash \match{x}{y}{e_1}{z}{e_2} : \tau}
  \\
  \inferrule*[right=$\land$R]
  {\G \vdash e_1 : \tau_1 \and \G \vdash e_2 : \tau_2}
  {\G \vdash \epair{e_1}{e_2} : \tau_1 \land \tau_2}
  \and
  \inferrule*[right=$\land$L]
  {\G, y : \tau_1, z : \tau_2 \vdash e : \tau}
  {\G, x : \tau_1 \land \tau_2 \vdash \letpair{y}{z}{x}{e} : \tau}
  \and
  \inferrule*[right=$\Rightarrow$R]
  {\G, x : \tau_1 \vdash e : \tau_2}
  {\G \vdash \elam{x}{e} : \tau_1 \Rightarrow \tau_2}
  \and
  \inferrule*[right=$\Rightarrow$L]
  {\G \vdash e_1 : \tau_1 \and \G, x : \tau_2 \vdash e : \tau}
  {\G, x : \tau_1 \Rightarrow \tau_2 \vdash \letapp{y}{f}{e_1}{e_2} : \tau}
  \end{mathpar}

\subsection*{Linear Logic}

\paragraph*{Syntax}
\begin{align*}
  \text{Expressions} \quad P, Q & ::= \esendl{x}{\m{inl}} \semi P \mid \esendl{x}{\m{inr}} \semi P \mid \case{x}{\m{inl} \Rightarrow P \mid \m{inr} \Rightarrow Q} \\
  & \;\; \mid \;\; \esend{x}{y} \semi P \mid \erecv{x}{y} \semi P \mid \ewait{x} \semi P \mid \eclose{x}\\
  \text{Types} \quad A, B & ::= A \oplus B \mid A \with B \mid A \tensor B \mid A \lolli B \mid \one
\end{align*}

\paragraph*{Typing Rules}
  \begin{mathpar}
  \inferrule*[right=$\oplus$R$_1$]
  {\Delta \vdash P :: (x : A_1)}
  {\Delta \vdash (x.\m{inl} \semi P) :: (x : A_1 \oplus A_2)}
  \and
  \inferrule*[right=$\oplus$R$_2$]
  {\Delta \vdash P :: (x : A_2)}
  {\Delta \vdash (x.\m{inr} \semi P) :: (x : A_1 \oplus A_2)}
  \and
  \inferrule*[right=$\oplus$L]
  {\Delta, x : A_1 \vdash Q_1 :: (z : C) \and \Delta, x : A_2 \vdash Q_2 :: (z : C)}
  {\Delta, x : A_1 \oplus A_2 \vdash (\ecaselr{x}{Q_1}{Q_2}) :: (z : C)}
  \and
  \inferrule*[right=$\with$R]
  {\Delta \vdash P_1 :: (x : A_1) \and \Delta \vdash P_2 :: (x : A_2)}
  {\Delta \vdash (\ecaselr{x}{P_1}{P_2}) :: (x : A_1 \with A_2)}
  \and
  \inferrule*[right=$\with$L$_1$]
  {\Delta, x : A_1 \vdash Q :: (z : C)}
  {\Delta, x : A_1 \with A_2 \vdash (x.\m{inl} \semi Q) :: (z : C)}
  \and
  \inferrule*[right=$\with$L$_2$]
  {\Delta, x : A_2 \vdash Q :: (z : C)}
  {\Delta, x : A_1 \with A_2 \vdash (x.\m{inr} \semi Q) :: (z : C)}
\end{mathpar}
    
\begin{mathpar}
  \inferrule*[right=$\tensor$R]
  {\Delta \vdash P :: (x : B)}
  {\Delta, y : A \vdash (\esend{x}{y} \semi P) :: (x : A \tensor B)}
  \and
  \inferrule*[right=$\tensor$L]
  {\Delta, y : A, x : B \vdash Q :: (z : C)}
  {\Delta, x : A \tensor B \vdash (\erecv{x}{y} \semi Q) :: (z : C)}
  \and
  \inferrule*[right=$\lolli$R]
  {\Delta, y : A \vdash P :: (x : B)}
  {\Delta \vdash (\erecv{x}{y} \semi P) :: (x : A \lolli B)}
  \and
  \inferrule*[right=$\lolli$L]
  {\Delta, x : B \vdash Q :: (z : C)}
  {\Delta, x : A \lolli B, y : A \vdash (\esend{x}{y} \semi Q) :: (z : C)}
  \and
  \inferrule*[right=$\one$R]
  {\mathstrut}
  {\cdot \vdash (\eclose{x}) :: (x : \one)}
  \and
  \inferrule*[right=$\one$L]
  {\Delta \vdash Q :: (z : C)}
  {\Delta, x : \one \vdash (\ewait{x} \semi Q) :: (z : C)}
  \and
  \inferrule*[right=$\m{id}$]
  { }
  {x : A \vdash (\fwd{y}{x}) :: (y : A)}
  \and
  \inferrule*[right=$\m{def}$]
  {\m{decl} \; f : \overline{y' : A'} \vdash (x : A) \in \Sigma
  \and \Delta, x : A \vdash Q :: (z : C)}
  {\Delta, \overline{y : A'} \vdash (\espawn{x}{f}{\overline{y}} \semi Q) :: (z : C)}
  \end{mathpar}

\section{Inference in Constructive Logic [50 pts]}
For the following propositions, determine if they are true or false.
If the proposition is true, do a (single) derivation with the corresponding program
(using the rules above).
If the proposition is false, briefly explain why.

\begin{enumerate}
  \item $(\tau_1 \land (\tau_2 \lor \tau_3)) \Rightarrow (\tau_1 \land \tau_2) \lor (\tau_1 \land \tau_3)$
  \item $((\tau_1 \Rightarrow \tau_2) \Rightarrow \tau_3) \Rightarrow (\tau_1 \Rightarrow (\tau_2 \Rightarrow \tau_3))$
  \item $\tau \Rightarrow (\tau \land \tau)$
  \item $(\tau \land \tau) \Rightarrow \tau$
  \item $\tau_1 \Rightarrow \tau_2 \Rightarrow (\tau_1 \land \tau_2)$
\end{enumerate}

\begin{solution}
  
\end{solution}

\section{Inference in Linear Logic [50 pts]}
For the following propositions, determine if they are true or false.
If the proposition is true, do a (single) derivation with the corresponding program
(using the rules above).
If the proposition is false, briefly explain why.

\begin{enumerate}
  \item $A \with (B \tensor C) \Rightarrow (A \with B) \tensor (A \with C)$
  \item $((A \lolli B) \lolli C) \Rightarrow (A \lolli (B \lolli C))$
  \item $(A \lolli A) \lolli A$
  \item $A \lolli A \lolli (A \tensor A)$
  \item $(A \lolli (B \tensor C)) \lolli (A \lolli B) \tensor (A \lolli C)$
\end{enumerate}

\begin{solution}
  
\end{solution}


\end{document}