\documentclass{article}

% Language setting
% Replace `english' with e.g. `spanish' to change the document language
\usepackage[english]{babel}

% Set page size and margins
% Replace `letterpaper' with`a4paper' for UK/EU standard size
\usepackage[letterpaper,top=2cm,bottom=2cm,left=3cm,right=3cm,marginparwidth=1.75cm]{geometry}

% Useful packages
\usepackage{amsmath}
\usepackage{amssymb}
\usepackage{graphicx}
\usepackage{mathpartir}
\usepackage{listings}
\lstset{basicstyle=\ttfamily}
\usepackage{courier}
\usepackage[colorlinks=true, allcolors=blue]{hyperref}

\title{CS 599 D1: Assignment 3}
\author{YOUR NAME, BU ID}
\date{}

\newcommand{\lc}{$\lambda$-calculus}

\newcommand{\m}[1]{\mathsf{#1}}
\newcommand{\mi}[1]{\mathit{#1}}
\newcommand{\mb}[1]{\mathbf{#1}}
\newcommand{\mt}[1]{\mathtt{#1}}
\newcommand{\elam}[2]{\lambda{#1}.\,{#2}}
\newcommand{\eapp}[2]{#1 \; #2}
\newcommand{\step}{\mapsto}
\newcommand{\val}[1]{#1 \; \m{value}}
\newcommand{\eval}{\Downarrow}
\newcommand{\num}[1]{\overline{#1}}
\newcommand{\numf}[1]{\underline{#1}}
\newcommand{\eif}[3]{\m{if} \; #1 \; \m{then} \; #2 \; \m{else} \; #3}
\newcommand{\multval}[1]{#1 \; \m{mult{-}value}}
\newcommand{\closed}[1]{#1 \; \m{closed}}
\newcommand{\tint}{\m{int}}
\newcommand{\tbool}{\m{bool}}
\newcommand{\elet}[3]{\m{let} \; #1 = #2 \; \m{in} \; #3}
\newcommand{\G}{\Gamma}
\newcommand{\R}[1]{\textcolor{red}{#1}}
\newcommand{\tfloat}{\m{float}}
\newcommand{\emptybox}{\boxed{\textcolor{white}{QWERTY}}}
\newcommand{\mstep}{\step^{*}}
\newcommand{\zero}{\m{zero}}
\renewcommand{\succ}[1]{\m{succ}(#1)}
\newcommand{\semi}{\,;\,}
\newcommand{\natrec}[5]{\m{natrec}(#1 \semi #2 \semi #3.\, #4. \, #5)}
\newcommand{\tnat}{\m{nat}}
\newcommand{\Red}{\m{Red}}
\newcommand{\natred}[1]{#1 \downarrow}
\newcommand{\issubst}{\Vdash}
\newcommand{\ift}{\m{int{-}or{-}float}}
\newcommand{\ecase}[3]{\m{case} \; #1 \; (#2 \Rightarrow #3)}
\newcommand{\case}[2]{\m{case} \; #1 \; (#2)}
\newcommand{\erecv}[2]{#2 \leftarrow \m{recv} \; #1}
\newcommand{\esend}[2]{\m{send} \; #1 \; #2}
\newcommand{\ewait}[1]{\m{wait} \; #1}
\newcommand{\eclose}[1]{\m{close} \; #1}
\newcommand{\fwd}[2]{#1 \leftrightarrow #2}
\newcommand{\espawn}[3]{#1 \leftarrow #2 \; #3}
\newcommand{\ichoice}[1]{\oplus\{#1\}}
\newcommand{\echoice}[1]{\&\{#1\}}
\newcommand{\with}{\,\&\,}
\newcommand{\one}{\mathbf{1}}
\newcommand{\tensor}{\otimes}
\newcommand{\lolli}{\multimap}

\newtheorem{problem}{Problem}
\newtheorem{theorem}{Theorem}
\newtheorem{lemma}{Lemma}
\newtheorem{definition}{Definition}
\newenvironment{solution}{\textbf{Solution.}}{}


\begin{document}
\maketitle


\begin{problem}[10 pts]
  Define a process called $\mt{increment}$ that has the following signature:
  \begin{lstlisting}
    decl increment: (x : bin) |- (y : bin)
  \end{lstlisting}
  The process uses a binary number $x$ as input and produces a binary number $y$ such that $y = x + 1$.
\end{problem}

\begin{solution}

\end{solution}

\begin{problem}[15 pts]
  First, define a $\mt{map}$ process with the following signature:
  \begin{lstlisting}
    decl map : (a : listA), (m : mapperAB) |- (b : listB)
  \end{lstlisting}
  The process uses a list $\mt{a : listA}$ and a mapper to produce a list $\mt{b : listB}$. For each element in $\mt{a}$,
  the $\mt{map}$ process sends the element of type $A$ to the mapper, receives an element of type $B$ which is then sent
  on the offered channel $\mt{b}$.
\end{problem}

\begin{solution}
  
\end{solution}

\begin{problem}[15 pts]
  Define a process called $\mt{mapinc}$ with the following signature:
  \begin{lstlisting}
    decl mapinc : . |- (m : binmapper)
  \end{lstlisting}
  This process receives binary numbers from $m$, increments them and sends them back to $m$.
  Remember that you have already defined the $\mt{increment}$ process for binary numbers.
  Be sure to call it to actually increment the binary number!
\end{problem}

\begin{solution}
  
\end{solution}

\begin{problem}[10 pts]
  Finally, define a process called $\mt{listinc}$ with the following signature:
  \begin{lstlisting}
    decl listinc : (a : binlist) |- (b : binlist)
  \end{lstlisting}
  This process uses a list $\mt{a : binlist}$ and increments each element to produce list $\mt{b : binlist}$.
  Instead of incrementing the elements on the list directly, use the $\mt{map}$ and $\mt{mapinc}$ processes 
  you defined earlier! You can assume the following type for the $\mt{map}$ process:
  \begin{lstlisting}
    decl map : (a : binlist), (m : binmapper) |- (b : binlist)
  \end{lstlisting}
\end{problem}

\begin{solution}
  
\end{solution}


\end{document}