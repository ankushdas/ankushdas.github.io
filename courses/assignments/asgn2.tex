\documentclass{article}

% Language setting
% Replace `english' with e.g. `spanish' to change the document language
\usepackage[english]{babel}

% Set page size and margins
% Replace `letterpaper' with`a4paper' for UK/EU standard size
\usepackage[letterpaper,top=2cm,bottom=2cm,left=3cm,right=3cm,marginparwidth=1.75cm]{geometry}

% Useful packages
\usepackage{amsmath}
\usepackage{amssymb}
\usepackage{graphicx}
\usepackage{mathpartir}
\usepackage{listings}
\lstset{basicstyle=\ttfamily}
\usepackage{courier}
\usepackage[colorlinks=true, allcolors=blue]{hyperref}

\title{CS 599 A1: Assignment 2}
\author{YOUR NAME, BU ID}
\date{}

\newcommand{\lc}{$\lambda$-calculus}

\newcommand{\m}[1]{\mathsf{#1}}
\newcommand{\mi}[1]{\mathit{#1}}
\newcommand{\mb}[1]{\mathbf{#1}}
\newcommand{\mt}[1]{\mathtt{#1}}
\newcommand{\elam}[2]{\lambda{#1}.\,{#2}}
\newcommand{\eapp}[2]{#1 \; #2}
\newcommand{\step}{\mapsto}
\newcommand{\val}[1]{#1 \; \m{value}}
\newcommand{\eval}{\Downarrow}
\newcommand{\num}[1]{\overline{#1}}
\newcommand{\numf}[1]{\underline{#1}}
\newcommand{\eif}[3]{\m{if} \; #1 \; \m{then} \; #2 \; \m{else} \; #3}
\newcommand{\multval}[1]{#1 \; \m{mult{-}value}}
\newcommand{\closed}[1]{#1 \; \m{closed}}
\newcommand{\tint}{\m{int}}
\newcommand{\tbool}{\m{bool}}
\newcommand{\elet}[3]{\m{let} \; #1 = #2 \; \m{in} \; #3}
\newcommand{\G}{\Gamma}
\newcommand{\R}[1]{\textcolor{red}{#1}}
\newcommand{\tfloat}{\m{float}}
\newcommand{\emptybox}{\boxed{\textcolor{white}{QWERTY}}}
\newcommand{\mstep}{\step^{*}}
\newcommand{\zero}{\m{zero}}
\renewcommand{\succ}[1]{\m{succ}(#1)}
\newcommand{\semi}{\,;\,}
\newcommand{\natrec}[5]{\m{natrec}(#1 \semi #2 \semi #3.\, #4. \, #5)}
\newcommand{\tnat}{\m{nat}}
\newcommand{\Red}{\m{Red}}
\newcommand{\natred}[1]{#1 \downarrow}
\newcommand{\issubst}{\Vdash}
\newcommand{\ift}{\m{int{-}or{-}float}}
\newcommand{\ecase}[3]{\m{case} \; #1 \; (#2 \Rightarrow #3)}
\newcommand{\case}[2]{\m{case} \; #1 \; (#2)}
\newcommand{\erecv}[2]{#2 \leftarrow \m{recv} \; #1}
\newcommand{\esend}[2]{\m{send} \; #1 \; #2}
\newcommand{\esendl}[2]{#1.#2}
\newcommand{\ewait}[1]{\m{wait} \; #1}
\newcommand{\eclose}[1]{\m{close} \; #1}
\newcommand{\fwd}[2]{#1 \leftrightarrow #2}
\newcommand{\espawn}[3]{#1 \leftarrow #2 \; #3}
\newcommand{\ichoice}[1]{\oplus\{#1\}}
\newcommand{\echoice}[1]{\&\{#1\}}
\newcommand{\with}{\,\&\,}
\newcommand{\one}{\mathbf{1}}
\newcommand{\tensor}{\otimes}
\newcommand{\lolli}{\multimap}
\newcommand{\config}{\mathcal{C}}
\newcommand{\dc}{\mathcal{D}}
\newcommand{\D}{\Delta}
\newcommand{\T}{\Theta}
\newcommand{\poised}[1]{#1 \; \m{poised}}
\newcommand{\proc}[2]{\m{proc}(#1, #2)}
\newcommand{\msg}[2]{\m{msg}(#1, #2)}
\newcommand{\fresh}[1]{(#1 \; \m{fresh})}

\newcommand{\cons}{\mathcal{C}}
\newcommand{\vars}{\mathcal{V}}
\newcommand{\proves}{\vDash}

\newcommand{\tforall}[1]{\forall #1. \, }
\newcommand{\texists}[1]{\exists #1. \, }
\newcommand{\tassertop}{{?}}  % -fp was: ?
\newcommand{\tassumeop}{{!}}  % -fp was: !
\newcommand{\tassert}[1]{\tassertop\{#1\}. \,} % -fp was: \; \tassertop
\newcommand{\tassume}[1]{\tassumeop\{#1\}. \,} % -fp was: \; \tassumeop

\newcommand{\esendn}[2]{\m{send} \; #1 \; \{#2\}}
\newcommand{\erecvn}[2]{\{#2\} \leftarrow \m{recv} \; #1}
\newcommand{\eassume}[2]{\m{assume} \; #1 \; \{#2\}}
\newcommand{\eassert}[2]{\m{assert} \; #1 \; \{#2\}}
\newcommand{\eimposs}{\m{impossible}}

\newtheorem{problem}{Problem}
\newtheorem{theorem}{Theorem}
\newtheorem{lemma}{Lemma}
\newtheorem{definition}{Definition}
\newenvironment{solution}{\textbf{Solution.}}{}


\begin{document}
\maketitle

\section{System T [60 pts]}
In this homework, we will study another popular language called G\"{o}del's System T that supports primitive recursion
(but importantly, not general recursion). The language is defined as follows:

\paragraph*{Syntax}
\begin{align*}
    \text{Expressions} \quad & e ::= \elam{x : \tau}{e} \mid \eapp{e}{e} \mid x \mid \zero \mid \succ{e} \mid \natrec{e}{e}{x}{y}{e} \\
    \text{Types} \quad & \tau ::= \tau \rightarrow \tau \mid \tnat
\end{align*}

The language has the standard expressions from \lc{} except that the $\lambda$-expression has the type of argument
stated explicitly.
In addition, the language natively supports natural numbers (not integers).
Natural numbers are defined inductively using (i) $\zero$ which defines the natural number $0$, and (ii) $\succ{e}$ which defines the
successor, i.e., if $e$ denotes the number $\num{n}$, then $\succ{e}$ denotes $\num{n+1}$.

\paragraph*{Type System}
\begin{mathpar}
    \inferrule*[right=Lam]
    {\G, x : \alpha \vdash e : \tau}
    {\G \vdash \elam{x : \alpha}{e} : \alpha \rightarrow \tau}
    \and
    \inferrule*[right=App]
    {\G \vdash e_1 : \alpha \rightarrow \tau \and \G \vdash e_2 : \alpha}
    {\G \vdash \eapp{e_1}{e_2} : \tau}
    \and
    \inferrule*[right=Var]
    { }
    {\G, x : \tau \vdash x : \tau}
    \and
    \inferrule*[right=Zero]
    { }
    {\G \vdash \zero : \tnat}
    \and
    \inferrule*[right=Succ]
    {\G \vdash e : \tnat}
    {\G \vdash \succ{e} : \tnat}
    \and
    \inferrule*[right=Rec]
    {\G \vdash e : \tnat \and \G \vdash e_0 : \tau \and \G, x: \tnat, y : \tau \vdash e_1 : \tau}
    {\G \vdash \natrec{e}{e_0}{x}{y}{e_1} : \tau}    
\end{mathpar}

The rules for \lc{} are standard. The type of $\zero$ is $\tnat$ and if $e$ has type $\tnat$,
then $\succ{e}$ also has type $\tnat$.
Finally, $\m{natrec}$ is used to perform primitive recursion on natural numbers (only). The expression
$\natrec{e}{e_0}{x}{y}{e_1}$ performs recursion on the first argument $e$ which must have type $\tnat$.
If $e$ is $\zero$, then this expression returns $e_0$ which has an arbitrary type $\tau$.
If $e$ is $\succ{e'}$, then we essentially evaluate $e_1$ where $x$ represents the predecessor $e'$,
and $y$ represents the value from the previous recursive call.
Therefore, in the presence of $x : \tnat$ and $y : \tau$, we need to derive that $e_1 : \tau$.


\paragraph*{Semantics}
\begin{mathpar}
    \inferrule*[right=$\lambda$-V]
    { }
    {\val{\elam{x : \tau}{e}}}
    \and
    \inferrule*[right=App-L]
    {e_1 \step e_1'}
    {\eapp{e_1}{e_2} \step \eapp{e_1'}{e_2}}
    \and
    \inferrule*[right=App-R]
    {\val{e_1} \and e_2 \step e_2'}
    {\eapp{e_1}{e_2} \step \eapp{e_1}{e_2'}}
    \and
    \inferrule*[right=App-S]
    {\val{e'}}
    {\eapp{(\elam{x : \tau}{e})}{e'} \step [e'/x]e}
    \and
    \inferrule*[right=Zero]
    { }
    {\val{\zero}}
    \and
    \inferrule*[right=Succ]
    { }
    {\val{\succ{e}}}
    \and
    \inferrule*[right=Rec-E]
    {e \step e'}
    {\natrec{e}{e_0}{x}{y}{e_1} \step \natrec{e}{e_0}{x}{y}{e_1}}
    \and
    \inferrule*[right=Rec-Z]
    { }
    {\natrec{\zero}{e_0}{x}{y}{e_1} \step e_0}
    \and
    \inferrule*[right=Rec-S]
    { }
    {\natrec{\succ{e}}{e_0}{x}{y}{e_1} \step [e/x,\, \natrec{e}{e_0}{x}{y}{e_1}/y]e_1}
\end{mathpar}

The rules of \lc{} are standard. $\zero$ and $\succ{e}$ are defined to be values.
There is a standard rule $\textsc{Rec-E}$ for evaluating the argument to $\m{natrec}$.
If the argument is $0$, the $\m{natrec}$ expression simply steps to $e_0$ as
demonstrated by the $\textsc{Rec-Z}$ rule.
Finally, if the argument is $\succ{e}$, then $e$ is substituted for $x$ and
the recursive call $\natrec{e}{e_0}{x}{y}{e_1}$ is substituted for $y$ in $e_1$.
\textbf{Try to write some examples of $e_0$ and $e_1$ to see how they evaluate.}

For your convenience, we will state (but not prove) a canonical forms lemma.

\begin{lemma}[Canonical Forms]
    If $\G \vdash e : \tau$, then
    \begin{itemize}
        \item If $\tau$ is $\tnat$, then $e = \zero$ or $e = \succ{e'}$ for some $e'$ such that $\G \vdash e' : \tnat$.
        \item If $\tau$ is $\tau_1 \rightarrow \tau_2$, then $e = \elam{x : \tau_1}{e'}$ for some $e'$ such that
        $\G, x : \tau_1 \vdash e' : \tau_2$.
    \end{itemize}
\end{lemma}

\subsection{Termination}
Unlike general recursive programming languages like OCaml and Rust, System T has the
valuable property that all programs written in this language terminate, i.e., evaluate to a value
in a finite number of steps.
Your task is to prove this fact using Tait's reducibility method.
The theorem we want to prove is the following:

\begin{theorem}[Normalization]\label{thm:normal}
    If $\cdot \vdash e : \tau$, then there exists $v$ such that $\val{v}$ and $e \mstep v$,
    where $\mstep$ is the reflexive transitive closure of $\step$.
\end{theorem}
We might hope to prove this
theorem directly by induction on the typing judgment. However, this approach is insufficient.
The case for the application rule $\textsc{(App)}$ is demonstrative.
\begin{mathpar}
    \inferrule*[right=App]
    {\G \vdash e_1 : \alpha \rightarrow \tau \and \G \vdash e_2 : \alpha}
    {\G \vdash \eapp{e_1}{e_2} : \tau}
\end{mathpar}
In this case, our induction hypotheses tells us that $e_1 \mstep v_1$ and $e_2 \mstep v_2$
for some values $v_1$ and $v_2$.
By preservation and the appropriate canonical forms lemma,
we know that $v_1 = \elam{x : \alpha}{e'}$ for some $e'$.
It also follows that $\eapp{e_1}{e_2} \mstep \eapp{v_1}{e_2}
\step [e_2/x]e'$.
Unfortunately, we are now stuck, as we have no
information about the behavior of $[e_2/x]e'$.

We will solve this by generalizing, proving a stronger statement which
gives us more information as an induction hypothesis. Specifically, we
will define a \emph{reducibility predicate} $\Red_{\tau}(e)$ and prove
the following theorem.

\begin{theorem}\label{thm:red}
    If $\cdot \vdash e : \tau$, then $\Red_{\tau}(e)$.
\end{theorem}
Since we'll define $\Red_\tau$ such that $\Red_\tau(e)$ implies the
existence of $\val{v}$ with $e \mstep v$, this theorem will imply
normalization as a corollary. The definition will go by structural
induction on the type $\tau$, which makes $\Red_\tau$ what is called a
\emph{logical relation}. (In particular, it is a \emph{unary} logical
relation; we will encounter \emph{binary} logical relations, such as
logical equivalence $e \sim_\tau e'$, later in the course.)  Actually,
we will prove an even more general theorem in order to account for
open terms; to state it concisely, we first want to define some
notation for substitutions.
\begin{definition}
  A substitution $\gamma = \{x_1 \hookrightarrow
  e_1, \ldots, x_n \hookrightarrow e_n\}$ is a finite mapping from
  variables to terms. Given an expression $e$, we write $\gamma(e)$
  for the expression $[e_1,\ldots,e_n/x_1,\ldots,x_n]e$, that is, the
  simultaneous substitution in $e$ of each expression $e_i$ for its
  corresponding variable $x_i$. For $\gamma$ as above, we define
  $\gamma \issubst \G$ to mean that $\G = x_1 : \tau_1, \ldots, x_n :
  \tau_n$ for some $\tau_1,\ldots,\tau_n$ such that
  $\Red_{\tau_i}(e_i)$ holds for $1 \le i \le n$.
\end{definition}
Now we state the theorem we will actually prove:
\begin{theorem}\label{thm:red_subst}
  If $\G \vdash e : \tau$ and $\gamma \issubst \G$ then $\Red_\tau(\gamma(e))$.
\end{theorem}
Theorem~\ref{thm:red} follows as the special case where $\G = \cdot$ and $\gamma =
\langle \rangle$. Finally, we define the predicate $\Red_\tau$ by
structural induction on $\tau$:
\begin{itemize}
  \item $\Red_{\tau_1 \to \tau_2}(e)$ holds if
    \begin{enumerate}
      \item $\cdot \vdash e : \tau_1 \to \tau_2$,
      \item there exists $\val{v}$ such that $e \mstep v$, and
      \item for any $e'$ such that $\Red_{\tau_1}(e')$, we have
        $\Red_{\tau_2}(e e')$.
    \end{enumerate}
  \item $\Red_\tnat(e)$ holds if
    \begin{enumerate}
      \item $\cdot \vdash e : \tnat$,
      \item there exists $\val{v}$ such that $e \mstep v$, and
      \item $\natred{v}$, where $\natred{v}$ is a judgment defined by
        \begin{mathpar}
        \inferrule*[right=$\downarrow$-Z]
          { }
          {\natred{\zero}}
        \and
        \inferrule*[right=$\downarrow$-S]
          {e \mstep v \and \val{v} \and \natred{v}}
          {\natred{\succ{e}}}
        \end{mathpar}
    \end{enumerate}
\end{itemize}
Note that $\Red_{\tau_1 \to \tau_2}(e)$ is defined in terms of $\Red$
at the structurally smaller types $\tau_1$ and $\tau_2$, so the
definition is well-founded. To get you started on the proof, and to
see how this definition succeeds where the previous attempt failed,
here is the $\textsc{App}$ case:
\begin{itemize}
  \item Case $\textsc{App}$: We have $\G \vdash \eapp{e_1}{e_2} : \tau$
    with $\G \vdash e_1 : \alpha \to \tau$ and $\G \vdash e_2 :
    \alpha$ for some $\alpha$. Per the theorem statement, we assume we
    are given $\gamma \issubst \G$ and want to prove that
    $\Red_\tau(\gamma(\eapp{e_1}{e_2}))$.  By definition of substitution, we
    have that $\gamma(\eapp{e_1}{e_2}) = \eapp{\gamma(e_1)}{\gamma(e_2)}$. Moreover,
    our induction hypotheses tell us that $\Red_{\alpha \to
      \tau}(\gamma(e_1))$ and $\Red_{\alpha}(\gamma(e_2))$. From
    condition 3 in the definition of $\Red_{\alpha \to \tau}$, we know
    that for any $e'$ with $\Red_{\alpha}(e')$ we have
    $\Red_{\tau}(\gamma(e_1)e')$. Taking $e' = \gamma(e_2)$ thus gives
    our goal.
\end{itemize}
With the right definition $\Red$ in hand, the $\textsc{App}$ case
follows almost trivially. On the other hand, the $\textsc{Lam}$ case
becomes more difficult. In general, though, proving the theorem is the
easy part of a logical relations argument -- the hard part is choosing
the right theorem to prove.

To complete the proof, you'll need the following lemma.
\begin{lemma}[Closure under Head Expansion]
  If $\Red_{\tau}(e')$,
  $\cdot \vdash e : \tau$ and $e \step e'$, then $\Red_{\tau}(e)$.
\end{lemma}

\begin{problem}[10 pts]
  Prove closure under head expansion.
\end{problem}

\begin{solution}

\end{solution}

With the help of Preservation, closure under head expansion extends to
apply when $e \mstep e'$ in multiple steps (you can use this without
proof).

\begin{problem}[30 pts]
  Prove the remaining cases of Theorem~\ref{thm:red_subst}.
  You may state (without proof) lemmas about substitution, but be sure
  to check that they are actually true.
\end{problem}

\begin{solution}

\end{solution}

\subsection{Programming in System T}

Next, we will get some programming experience in System T.

\begin{problem}[20 pts]
    Define the following functions in System T. For each definition below,
    briefly explain the intuition behind your answer (4 pts each).
    You can define and use helper functions to solve these problems.

    \begin{enumerate}
        \item Define $\m{mult}$, where $\m{mult} \; \num{m} \; \num{n} \mstep \num{m \otimes n}$.
        \item Define $\m{minus}$, where $\m{minus} \; \num{m} \; \num{n} \mstep \num{m - n}$ if $\num{m} > \num{n}$. It should produce 0 otherwise.
        \item Define $\m{leq}$, where $\m{leq} \; \num{m} \; \num{n} \mstep \succ{\zero}$ if $\num{m} \leq \num{n}$ and
        $\m{leq} \; \num{m} \; \num{n} \mstep \zero$ otherwise.
        \item Define $\m{mod}$ where $\m{mod} \; \num{m} \; \num{n} = \num{m \; \m{mod} \; n}$. You may pick appropriate defaults when $n = 0$.
        \item Define $\m{cube}$ where $\m{cube} \; \num{n} = \num{n \otimes n \otimes n}$.
    \end{enumerate}

\end{problem}


\begin{solution}
  
\end{solution}


\end{document}