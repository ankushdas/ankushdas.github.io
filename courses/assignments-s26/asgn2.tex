\documentclass{article}

% Language setting
% Replace `english' with e.g. `spanish' to change the document language
\usepackage[english]{babel}

% Set page size and margins
% Replace `letterpaper' with`a4paper' for UK/EU standard size
\usepackage[letterpaper,top=1in,bottom=1in,left=1in,right=1in]{geometry}

% Useful packages
\usepackage{amsmath}
\usepackage{amssymb}
\usepackage{enumerate}
\usepackage{graphicx}
\usepackage{mathpartir}
\usepackage[colorlinks=true, allcolors=blue]{hyperref}

\title{CS 599 A1: Assignment 2}
\author{YOUR NAME, BU ID}
\date{}

\newcommand{\lc}{$\lambda$-calculus}

\newcommand{\m}[1]{\mathsf{#1}}
\newcommand{\pt}{\; \m{true}}
\newcommand{\mi}[1]{\mathit{#1}}
\newcommand{\mb}[1]{\mathbf{#1}}
\newcommand{\mt}[1]{\mathtt{#1}}
\newcommand{\elam}[2]{\lambda{#1}.\,{#2}}
\newcommand{\eapp}[2]{#1 \; #2}
\newcommand{\step}{\mapsto}
\newcommand{\val}[1]{#1 \; \m{value}}
\newcommand{\eval}{\Downarrow}
\newcommand{\num}[1]{\overline{#1}}
\newcommand{\numf}[1]{\underline{#1}}
\newcommand{\eif}[3]{\m{if} \; #1 \; \m{then} \; #2 \; \m{else} \; #3}
\newcommand{\multval}[1]{#1 \; \m{mult{-}value}}
\newcommand{\closed}[1]{#1 \; \m{closed}}
\newcommand{\tint}{\m{int}}
\newcommand{\tbool}{\m{bool}}
\newcommand{\elet}[3]{\m{let} \; #1 = #2 \; \m{in} \; #3}
\newcommand{\G}{\Gamma}
\newcommand{\R}[1]{\textcolor{red}{#1}}
\newcommand{\tfloat}{\m{float}}
\newcommand{\emptybox}{\boxed{\textcolor{white}{QWERTY}}}
\newcommand{\mstep}{\step^{*}}
\newcommand{\zero}{\m{zero}}
\renewcommand{\succ}[1]{\m{succ}(#1)}
\newcommand{\semi}{\,;\,}
\newcommand{\natrec}[5]{\m{natrec}(#1 \semi #2 \semi #3.\, #4. \, #5)}
\newcommand{\tnat}{\m{nat}}
\newcommand{\Red}{\m{Red}}
\newcommand{\natred}[1]{#1 \downarrow}
\newcommand{\issubst}{\Vdash}
\newcommand{\ift}{\m{int{-}or{-}float}}
\newcommand{\ecase}[3]{\m{case} \; #1 \; (#2 \Rightarrow #3)}
\newcommand{\ecaselr}[3]{\m{case} \; #1 \; (\m{inl} \Rightarrow #2 \mid \m{inr} \Rightarrow #3)}
\newcommand{\case}[2]{\m{case} \; #1 \; (#2)}
\newcommand{\erecv}[2]{#2 \leftarrow \m{recv} \; #1}
\newcommand{\esend}[2]{\m{send} \; #1 \; #2}
\newcommand{\esendl}[2]{#1.#2}
\newcommand{\ewait}[1]{\m{wait} \; #1}
\newcommand{\eclose}[1]{\m{close} \; #1}
\newcommand{\fwd}[2]{#1 \leftrightarrow #2}
\newcommand{\espawn}[3]{#1 \leftarrow #2 \; #3}
\newcommand{\ichoice}[1]{\oplus\{#1\}}
\newcommand{\echoice}[1]{\&\{#1\}}
\newcommand{\with}{\,\&\,}
\newcommand{\one}{\mathbf{1}}
\newcommand{\tensor}{\otimes}
\newcommand{\lolli}{\multimap}
\newcommand{\config}{\mathcal{C}}
\newcommand{\dc}{\mathcal{D}}
\newcommand{\D}{\Delta}
\newcommand{\T}{\Theta}
\newcommand{\poised}[1]{#1 \; \m{poised}}
\newcommand{\proc}[2]{\m{proc}(#1, #2)}
\newcommand{\msg}[2]{\m{msg}(#1, #2)}
\newcommand{\fresh}[1]{(#1 \; \m{fresh})}
\newcommand{\up}{\,\uparrow^{\m{S}}_{\m{L}}}
\newcommand{\down}{\,\downarrow^{\m{S}}_{\m{L}}}

\newcommand{\cons}{\mathcal{C}}
\newcommand{\vars}{\mathcal{V}}
\newcommand{\proves}{\vDash}

\newcommand{\tforall}[1]{\forall #1. \, }
\newcommand{\texists}[1]{\exists #1. \, }
\newcommand{\tassertop}{{?}}  % -fp was: ?
\newcommand{\tassumeop}{{!}}  % -fp was: !
\newcommand{\tassert}[1]{\tassertop\{#1\}. \,} % -fp was: \; \tassertop
\newcommand{\tassume}[1]{\tassumeop\{#1\}. \,} % -fp was: \; \tassumeop

\newcommand{\esendn}[2]{\m{send} \; #1 \; \{#2\}}
\newcommand{\erecvn}[2]{\{#2\} \leftarrow \m{recv} \; #1}
\newcommand{\eassume}[2]{\m{assume} \; #1 \; \{#2\}}
\newcommand{\eassert}[2]{\m{assert} \; #1 \; \{#2\}}
\newcommand{\eimposs}{\m{impossible}}

\newcommand{\epair}[2]{\langle #1, #2 \rangle}
\newcommand{\letpair}[4]{\m{let} \; \langle #1, #2 \rangle = #3 \; \m{in} \; #4}
\newcommand{\inl}[1]{\m{inl}(#1)}
\newcommand{\inr}[1]{\m{inr}(#1)}
\newcommand{\match}[5]{\m{match} \; #1 \; \m{with} \; \{\,\m{inl}(#2) \to #3 \mid \m{inr}(#4) \to #5\,\}}
\newcommand{\letapp}[4]{\m{let} \; #1 = #2 \; #3 \; \m{in} \; #4}

\newcommand{\limpl}{\supset}
\newcommand{\Nat}{\mathbb{N}}
\newcommand{\Real}{\mathbb{R}}

\newtheorem{problem}{Problem}
\newtheorem{theorem}{Theorem}
\newtheorem{lemma}{Lemma}
\newtheorem{definition}{Definition}
\newenvironment{solution}{\textbf{Solution.}}{}

\makeatletter
\def\arcr{\@arraycr}
\makeatother


\begin{document}
\maketitle

\begin{problem}[Mandatory; No LLM; 10 pts]
    Recall the natural deduction problems from Assignment 1.
    For the following propositions, provide
    the corresponding proof term \emph{[2.5 pts each]}.

    \begin{enumerate}[(i)]
        \item $(A \limpl (B \limpl C)) \limpl (A \land B) \limpl C$
        \item $A \limpl \neg \neg A$
        \item $(\neg A \lor \neg B) \limpl \neg (A \land B)$
        \item $(A \land (B \lor C)) \limpl (A \land B) \lor (A \land C)$
    \end{enumerate}
\end{problem}

\begin{solution}
    YOUR SOLUTION GOES HERE
\end{solution}


\begin{problem}[Mandatory; No LLM; 15 pts]
Recall the $A \land B$ operator from constructive logic. Suppose we keep the introduction rule unchanged, but update the
elimination rule as follows:
    \begin{mathpar}
        \inferrule*[right=$\land I$]
        {A \pt \and B \pt}
        {A \land B \pt}
        \and
        \inferrule*[right=$\land E^{x,y}$]
        {
        \begin{array}{cc}
            & \;\;\;\;\inferrule*[right=$x$]
            { }
            {A \pt}
            \quad
            \inferrule*[right=$y$]
            { }
            {B \pt} \arcr
            & \vdots \arcr
            A \land B \pt &
            \inferrule*
            {}
            {C \pt}
        \end{array}    
        }
        {C \pt}
    \end{mathpar}
\begin{itemize}
    \item Are the rules locally sound? If yes, show a reduction. Otherwise, give an explanation of unsoundness.
    \item Are the rules locally complete? If yes, show an expansion. Otherwise, give an explanation of incompleteness.
    \item Provide an appropriate proof term for the elimination rule.
\end{itemize}
\end{problem}

\begin{solution}
    YOUR SOLUTION GOES HERE
\end{solution}

\begin{problem}[Mandatory; No LLM; 15 pts]
    Recall the $A \limpl B$ operator from constructive logic. Suppose we keep the introduction rule unchanged, but update the
    elimination rule as follows:
    \begin{mathpar}
        \inferrule*[right=$\limpl I^x$]
        {\begin{array}{c}
            \;\;\;\inferrule*[right=$x$]
            { }
            {A \pt} \arcr
            \vdots \arcr
            B \pt
        \end{array}}
        {A \limpl B \pt}
        \and
        \inferrule*[right=$\limpl E^{x}$]
        {
        \begin{array}{cc}
            & \;\;\;\inferrule*[right=$x$]
            { }
            {B \pt} \arcr
            & \vdots \arcr
            A \limpl B \pt \and A \pt &
            \inferrule*
            {}
            {C \pt}
        \end{array}    
        }
        {C \pt}
    \end{mathpar}
\begin{itemize}
    \item Are the rules locally sound? If yes, show a reduction. Otherwise, give an explanation of unsoundness.
    \item Are the rules locally complete? If yes, show an expansion. Otherwise, give an explanation of incompleteness.
    \item Provide an appropriate proof term for the elimination rule.
\end{itemize}
\end{problem}

\begin{solution}
    YOUR SOLUTION GOES HERE
\end{solution}

\begin{problem}[Mandatory; No LLMs; 15 pts]
Suppose we define a new operator $A \star B$ with the following introduction and elimination rules:
\begin{mathpar}
    \inferrule*[right=$\star I_1^x$]
    {\begin{array}{c}
        \;\;\;\inferrule*[right=$x$]
        { }
        {A \pt} \arcr
        \vdots \arcr
        B \pt
    \end{array}}
    {A \star B \pt}
    \and
    \inferrule*[right=$\star I_2^x$]
    {\begin{array}{c}
        \;\;\;\inferrule*[right=$x$]
        { }
        {B \pt} \arcr
        \vdots \arcr
        A \pt
    \end{array}}
    {A \star B \pt}
    \and
    \inferrule*[right=$\star E^{x,y}$]
    {
    \begin{array}{ccccc}
        & & \;\;\;\inferrule*[right=$x$]
        { }
        {B \pt} & &
        \;\;\;\inferrule*[right=$y$]
        { }
        {A \pt} \arcr
        & & \vdots & & \vdots \arcr
        A \star B \pt & \quad A \pt &
        \inferrule*
        {}
        {C \pt}
        & B \pt
        & C \pt
    \end{array}    
    }
    {C \pt}
\end{mathpar}
    \begin{itemize}
        \item Are the rules locally sound? If yes, show a reduction. Otherwise, give an explanation of unsoundness.
        \item Are the rules locally complete? If yes, show an expansion. Otherwise, give an explanation of incompleteness.
        \item Can $A \star B$ be expressed using the standard operators in constructive logic?
    \end{itemize}
\end{problem}

\begin{solution}
    YOUR SOLUTION GOES HERE
\end{solution}

\begin{problem}[Mandatory; No LLMs; 15 pts]
Suppose we define a new operator $A \heartsuit B \Diamond C$ with the following introduction and elimination rules:
\begin{mathpar}
    \inferrule*[right=$\heartsuit \Diamond I_1$]
    {A \pt}
    {A \heartsuit B \Diamond C \pt}
    \and
    \inferrule*[right=$\heartsuit \Diamond I_2$]
    {B \pt \and C \pt}
    {A \heartsuit B \Diamond C \pt}
    \and
    \inferrule*[right=$\heartsuit \Diamond E^{x,y,z}$]
    {
    \begin{array}{cccc}
        & \;\;\;\inferrule*[right=$x$]
        { }
        {B \pt}
        \quad
        \inferrule*[right=$y$]
        { }
        {C \pt} & &
        \;\;\;\inferrule*[right=$z$]
        { }
        {A \pt} \arcr
        & \vdots & & \vdots \arcr
        A \heartsuit B \Diamond C \pt &
        \inferrule*
        {}
        {D \pt} &
        & D \pt
    \end{array}    
    }
    {D \pt}
\end{mathpar}
    \begin{itemize}
        \item Are the rules locally sound? If yes, show a reduction. Otherwise, give an explanation of unsoundness.
        \item Are the rules locally complete? If yes, show an expansion. Otherwise, give an explanation of incompleteness.
        \item Provide a derivation of $A \heartsuit B \Diamond C \limpl A \heartsuit C \Diamond B \pt$.
    \end{itemize}
\end{problem}

\begin{solution}
    YOUR SOLUTION GOES HERE
\end{solution}

\begin{problem}[Optional; No LLMs; 15 pts]
Determine if the following propositions are derivable using rules of verifications ($A \uparrow$) and uses ($A \downarrow$).
If yes, provide a derivation. Otherwise, briefly explain why.
\begin{enumerate}[(i)]
    \item $((A \lor B) \land (A \limpl C) \land (B \limpl C)) \limpl C \uparrow$ 
    \item $(A \limpl B) \limpl ((A \lor C) \limpl (B \lor C)) \uparrow$
    \item $(A \lor B) \limpl ((A \limpl C) \limpl ((B \limpl D) \limpl (C \lor D))) \uparrow$
\end{enumerate}
\end{problem}

\begin{solution}
    YOUR SOLUTION GOES HERE
\end{solution}

\begin{problem}[Optional; LLMs ok; 30 pts]
Given a derivation $\mathcal{D}$ of $A \; \m{true}$, can you construct a derivation $\mathcal{E}$
of $A \uparrow$. \\
Hint: Try this problem with only the $A \land B$ and $A \limpl B$ operators first.
\end{problem}

\begin{solution}
    YOUR SOLUTION GOES HERE
\end{solution}

\begin{problem}[Optional; LLMs ok; 15 pts]
Design introduction and elimination rules of a new connective $A \bullet B$ such that it is
locally sound but not locally complete.
\end{problem}

\begin{solution}
    YOUR SOLUTION GOES HERE
\end{solution}

\begin{problem}[Optional; LLMs ok; 15 pts]
Design introduction and elimination rules of a new connective $A \circ B$ such that it is
locally complete but not locally sound.
\end{problem}

\begin{solution}
    YOUR SOLUTION GOES HERE
\end{solution}


\end{document}